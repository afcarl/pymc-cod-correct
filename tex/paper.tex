%%%%%%%%%%%%%%%%%%%%%%%%%%%%%%%%%%%%%%%%%%%%%%
%%                                          %%
%% The Abstract begins here                 %%
%%                                          %%
%% The Section headings here are those for  %%
%% a Research article submitted to a        %%
%% BMC-Series journal.                      %%  
%%                                          %%
%% If your article is not of this type,     %%
%% then refer to the Instructions for       %%
%% authors on http://www.biomedcentral.com  %%
%% and change the section headings          %%
%% accordingly.                             %%   
%%                                          %%
%%%%%%%%%%%%%%%%%%%%%%%%%%%%%%%%%%%%%%%%%%%%%%
%% The abstract of the manuscript should not exceed 350 words and must
%% be structured into separate sections: Background, the context and
%% purpose of the study; Methods, how the study was performed and
%% statistical tests used; Results, the main findings; Conclusions,
%% brief summary and potential implications. Please minimize the use
%% of abbreviations and do not cite references in the abstract; Trial
%% registration, if your research article reports the results of a
%% controlled health care intervention, please list your trial
%% registry, along with the unique identifying number, e.g. Trial
%% registration: Current Controlled Trials ISRCTN73824458. Please note
%% that there should be no space between the letters and numbers of
%% your trial registration number.

\begin{abstract}
        % Do not use inserted blank lines (ie \\) until main body of text.
        \paragraph*{Background:} Population-level information on cause-specific mortality rates by country, age, sex, and time is important for decision-making.  Models are crucial to estimating and predicting this information when direct measurements are not available.  State-of-the-art models produce estimates disease by disease, however, and do not enforce the constraint that the sum of deaths over every cause must be equal to the total deaths due to all causes.
      
        \paragraph*{Methods:} We developed a statistical model to correct for inconsistencies between cause-specific mortality rate estimates and all-cause mortality estimates. We compared our model to a naive approach in a validation environment using a series of simulation studies designed to test TK.  We measured the quality of the corrected estimates by TK.

        \paragraph*{Results:} Over all simulations, our approach performed TK, compared to TK for the naive approach.  For setting TK, our approach showed significant benefits in terms of TK and TK.  For setting TK, neither approach performed very well, demonstrating that without sufficiently accurate input data, all model-based approaches have limitations.

        \paragraph*{Conclusions:} Our model is a fast and effective way to improve the estimates produced by many separate models, and we recommend using it to enforce consistency between cause-specific mortality rates and all-cause mortality rates.
\end{abstract}



\ifthenelse{\boolean{publ}}{\begin{multicols}{2}}{}




%%%%%%%%%%%%%%%%%%%%%%%%%%%%%%%%%%%%%%%%%%%%%%
%%                                          %%
%% The Main Body begins here                %%
%%                                          %%
%% The Section headings here are those for  %%
%% a Research article submitted to a        %%
%% BMC-Series journal.                      %%  
%%                                          %%
%% If your article is not of this type,     %%
%% then refer to the instructions for       %%
%% authors on:                              %%
%% http://www.biomedcentral.com/info/authors%%
%% and change the section headings          %%
%% accordingly.                             %% 
%%                                          %%
%% See the Results and Discussion section   %%
%% for details on how to create sub-sections%%
%%                                          %%
%% use \cite{...} to cite references        %%
%%  \cite{koon} and                         %%
%%  \cite{oreg,khar,zvai,xjon,schn,pond}    %%
%%  \nocite{smith,marg,hunn,advi,koha,mouse}%%
%%                                          %%
%%%%%%%%%%%%%%%%%%%%%%%%%%%%%%%%%%%%%%%%%%%%%%

%%%%%%%%%%%%%%%%
%% Background %%
%%
%% The background section should be written from the standpoint of
%% researchers without specialist knowledge in that area and must
%% clearly state - and, if helpful, illustrate - the background to the
%% research and its aims. Reports of clinical research should, where
%% appropriate, include a summary of a search of the literature to
%% indicate why this study was necessary and what it aimed to
%% contribute to the field. The section should end with a very brief
%% statement of what is being reported in the article.

\section*{Background}
  TK one paragraph on population health metrics, GBD, CoD paper which
  uses different ensemble models for each cause and how hard it would
  be to combine them.

  Models for each cause are estimated separately, so there is no
  guarantee of consistency and the estimated cause fractions for any
  given country-sex-year do not necessarily sum to one.

  TK one paragraph on what we have done, a model-based approach to
  enforcing consistency between cause-specific model estimates and
  all-cause estimates.  We validated our approach through a simulation
  study, with an environment constructed to capture important
  characteristics of the real challenge.  We propose two models
  that take different approaches to predicting consistent cause
  fractions from inconsistent preliminary estimates.  These two models
  are then validated in the simulation environment and compared on a
  variety of metrics.

  TK one paragraph on the importance of what we have done.

%%%%%%%%%%%%%%%%%%
%% Methods
%%
%% This should include the design of the study, the setting, the type
%% of participants or materials involved, a clear description of all
%% interventions and comparisons, and the type of analysis used,
%% including a power calculation if appropriate.

\section*{Methods}
The estimation challenge we faced was the following: For a given sex, age group,
and geographic region, we obtained estimates of the cause-specific
mortality fraction $F_j$ for a mutually exclusive, collectively
exhaustive list causes $j = 1,\ldots,J$, using a statistical model described elsewhere \ref{TK}.  Each estimate is
represented by $N$ draws from the posterior distribution over values
for $T$ years:
\[
F_j = \left(f_{j,y_1}^n, \ldots, f_{j,y_T}^n\right)_{n=1}^N.
\]
Below we have also used the notation $F_{j,y}$ to denote the marginal empirical distribution of $F_j$ for year $y$:
\[
F_{j,y} = \left(f_{j,y}^1, \ldots, f_{j,y}^N\right).
\]

Our goal was to produce a posterior distribution for the true cause
fractions $\pi_{j,y}$, by incorporating the consistency condition
$\sum_{j} \pi_{j,y}=1$ for all $y$.  This would allow us to produce
point estimates and uncertainty intervals for individual cause
fractions, as well as estimate the change in $\pi_{j,y}$ as a function
of $y$ for each $j$.

  \subsection*{Quality Metrics}
  We wanted the estimated cause fractions from any model to be close
  to the true cause fractions in terms of absolute error, although being
  accurate in terms of relative error is also important for some applications.
  Following the approach taken in Verbal Autopsy quality metrics
  \ref{VAMetrics}, we summarize the absolute error using cause
  specific mortality fraction accuracy (CSMF accuracy), defined by the
  following formula:
  \begin{align*}
    \text{CSMF Accuracy} = 1- \sum \frac{\left | \pi_{j,y} - \pi^{\text{true}}_{j,y} \right |}{2(1-\min_j(\pi^{\text{true}}_{j,y}))}
  \end{align*}

  We also wanted the estimates to have accurate uncertainty intervals,
  which we quantified in terms of coverage probability (the
  probability that the true estimate falls between the upper and lower
  bounds of the $95\%$ highest probability density (HPD) interval).  We desired that the
  coverage probability for each cause be close to $0.95$.

  TK description of bias, another important metric of estimation quality:
  \[
  \text{Mean Bias} = \sum_{j=1,\ldots, J} \left\{ \pi_{j,y} - \pi^{\text{true}}_{j,y} \right\}
  \]

  TK description of trend accuracy, a newly developed test statistic,
  designed to capture how accurately our model has captured the time
  trend of the corrected cause fraction estimates:
  \[
  \text{Trend Accuracy} = TK
  \]

  \subsection*{Statistical Models}
  Before developing a sophisticated statistical model for estimating
  $\pi_{j,y}$, we considered a naive approach: for each
  $n=1,\ldots,N$, for each $y$, we simply scaled the vector
  $(f_{1,y}^n, \ldots, f_{J,y}^n)$ to make it sum to unity:
  \begin{align*}
    \pi_{j,y}^n = \frac{f_{j,y}^n}{\sum_{j'=1}^J f_{j',y}^n}.
  \end{align*}
  This provided a non-parametric posterior distribution for
  $\pi_{j,y}$, which we summarized as a point estimate for each $j$
  and $y$ by taking the mean over values of $n$, and estimates
  uncertainty intervals by taking the $95\%$ HPD interval.

  We developed the more traditional statistical model that we now
  describe to take into account the differences in uncertainty that
  the different $f_{j,y}$ in the input data often exhibit.  The formal
  specification of the model is the following:
  \begin{align*}
    \dens(F_{j,y} | \pi) &\propto \exp\left\{-\frac{\left(\pi_{j,y}-\E[F_{j,y}]\right)^2}{\Var[F_{j,y}]} \right\}\\
    \pi_{j,y} &= \frac{\alpha_{j,y}^t}{\sum_{j'=1}^J \alpha_{j',y}^t}\\
    \alpha_{j,y} &\sim \Normal\left(0, 1^2\right)
  \end{align*}

  Here the notation $\E[F_{j,y}]$ and $\Var[F_{j,y}]$ are the usual mean and variance:
  \begin{align*}
    \E[F_{j,y}] &= \bigg(\sum_{n=1}^N f_{j,y}^n\bigg) / N,\\
    \Var[F_{j,y}] &= \bigg(\sum_{n=1}^N \left(f_{j,y}^n - \E[F_{j,y}]\right)^2\bigg) / N.\\
  \end{align*}

 This approach can be modified to explicitly account for time-correlation in the
 $F_j$ distribution, by making the likelihood look more like a
 multivariate normal:
  \begin{align*}
    \dens\left(F_j | \pi \right)
    &\propto \exp\left\{-\left(\pi_j - \E[F_j]\right)^T \Sigma_j^{-1} \left(\pi_j - \E[F_j]\right) \right\}\\
    \pi_{j,y} &= \frac{\alpha_{j,y}^t}{\sum_{j'=1}^J \alpha_{j',y}^t}\\
    \alpha_{j,y} &\sim \Normal\left(0, 1^2\right)\\
    \Sigma_j &= \Matern(\sigma, \rho_j, 2) + \Diag(\Var[F_j]))
    \sigma &\sim \Uniform[0,1]\\
    \rho_j &\sim \Normal_{\epsilon+}(20, 10^2)
  \end{align*}

  Here $\E[F_j]$ and $\Var[F_j]$ are the vector-valued analogues of
  the expectation and variance above, and $\Matern = TK$. TK more
  description of the Matern covariance.


  \subsection*{Simulation Environment}
  To validate the models above, and to compare the sophisticated
  approaches to the naive approach, and to generally assess how much
  improvement this sort of consistenty constraint can be expected to
  provide, we conducted a simulation study.  We chose true starting
  cause fractions $\pi^\true$ for a list of 10 causes from a heavy-tailed
  distribution TK, and then randomly perturbed them to vary smoothly
  over a 10 year time period, according to a Matern covariance
  function TK.  Then we used simulation to generate 1000 samples of
  noisy ``estimates'' of the true cause fractions to produce simulated
  input data for which ground truth was known.  We varied the
  dispersion of the noisy estimates, as well as the bias to understand
  how the different models perform in different settings.  We also
  varied the correlation between the bias and the variance, to explore
  how inappropriately narrow uncertainty in the CoD model would affect
  the corrected estimates.  This can be understood as a three-armed
  factorial study design, with dispersion taking values low, medium,
  and high, and bias taking values low, medium, and high, and
  correlation taking values of positive, zero, and negative.

  To be precise, the true mean cause fractions were generated by a
  normalized exponential transform of 10 Gaussian Processes with
  covariance $\Matern(TK)$, evaluated at times $1,2,\ldots,10$:
  \begin{align*}
  \alpha^{\true}_j(y) &\sim \GP(0, \Matern(TK)),\\
  \pi^{\true}_{j,y} &= \frac{\alpha_j(y)}{\sum_{j'=1}^J \alpha_{j'}(y)}.
  \end{align*}
  Then estimate of cause fractions were generated for draw $n$ from a two stage perturbation as follows:
  \begin{align*}
  \logit f^n_{j,y} &= \logit(\pi^{\true}_{j,y} + \beta_{j,y} + \epsilon_{j,y,n}),\\
  \beta_{j,y} &\sim \Normal(0, \sigma_\text{bias}^2),\\
  \epsilon_{j,y,n}|\beta_{j,y} &\sim \Normal(TK, \Sigma^2),\\
  \Sigma^2 = TK.matrix.with \sigma_\text{bias}^2, \sigma_\text{dispersion}^2, \rho_\text{correlation}.
  \end{align*}
  This results in $N$ draws from a distribution with the same bias and dispersion.

  We repeated this procedure $1000$ times for each of the $27$
  possibilities in the design, and calculated CSMF accuracy, coverage
  probability, and TK something about change over time for all of
  them.

%%%%%%%%%%%%%%%%%%%%%%%%%%%%
%% Results  %%
%%
%% The Results and Discussion may be combined into a single section or
%% presented separately. Results of statistical analysis should
%% include, where appropriate, relative and absolute risks or risk
%% reductions, and confidence intervals. The results and discussion
%% sections may also be broken into subsections with short,
%% informative headings.

\section*{Results/Discussion}
  \subsection*{Results sub-heading}

  \subsection*{Another results sub-heading}

  \subsection*{Yet another results sub-heading}

%%%%%%%%%%%%%%%%%%%%%%
%% This should state clearly the main conclusions of the research and
%% give a clear explanation of their importance and relevance. Summary
%% illustrations may be included.

\section*{Conclusions}
The good model is better than the bad model! Hurray!!! \pb
  
%%%%%%%%%%%%%%%%%%%%%%
%% If abbreviations are used in the text, either they should be
%% defined in the text where first used, or a list of abbreviations
%% can be provided, which should precede the competing interests and
%% authors' contributions.

\section*{List of abbreviations used}
    
%%%%%%%%%%%%%%%%%%%%%%%%%%%%%%%%
%% A competing interest exists when your interpretation of data or
%% presentation of information may be influenced by your personal or
%% financial relationship with other people or organizations. Authors
%% should disclose any financial competing interests but also any
%% non-financial competing interests that may cause them embarrassment
%% were they to become public after the publication of the manuscript.
%%
%% Authors are required to complete a declaration of competing
%% interests. All competing interests that are declared will be listed
%% at the end of published articles. Where an author gives no
%% competing interests, the listing will read 'The author(s) declare
%% that they have no competing interests'.
%%
%% When completing your declaration, please consider the following
%% questions:
%%
%% Financial competing interests
%%
%% * In the past five years have you received reimbursements, fees,
%%   funding, or salary from an organization that may in any way gain or
%%   lose financially from the publication of this manuscript, either
%%   now or in the future? Is such an organization financing this
%%   manuscript (including the article-processing charge)? If so, please
%%   specify.
%% * Do you hold any stocks or shares in an organization that may in
%%   any way gain or lose financially from the publication of this
%%   manuscript, either now or in the future? If so, please specify.
%% * Do you hold or are you currently applying for any patents
%%   relating to the content of the manuscript? Have you received
%%   reimbursements, fees, funding, or salary from an organization
%%   that holds or has applied for patents relating to the content of
%%   the manuscript? If so, please specify.
%% * Do you have any other financial competing interests? If so,
%%   please specify.
%%
%% Non-financial competing interests
%%
%% * Are there any non-financial competing interests (political,
%%   personal, religious, ideological, academic, intellectual,
%%   commercial or any other) to declare in relation to this
%%   manuscript? If so, please specify.
%%
%% If you are unsure as to whether you or one of your co-authors has a
%% competing interest, please discuss it with the editorial office.

\section*{Competing interests }

%%%%%%%%%%%%%%%%%%%%%%%%%%%%%%%%
%% 
%% In order to give appropriate credit to each author of a paper, the
%% individual contributions of authors to the manuscript should be
%% specified in this section.

%% An ``author'' is generally considered to be someone who has made
%% substantive intellectual contributions to a published study. To
%% qualify as an author one should 1) have made substantial
%% contributions to conception and design, or acquisition of data, or
%% analysis and interpretation of data; 2) have been involved in
%% drafting the manuscript or revising it critically for important
%% intellectual content; and 3) have given final approval of the
%% version to be published. Each author should have participated
%% sufficiently in the work to take public responsibility for
%% appropriate portions of the content. Acquisition of funding,
%% collection of data, or general supervision of the research group,
%% alone, does not justify authorship.

%% We suggest the following kind of format (please use initials to
%% refer to each author's contribution): AB carried out the molecular
%% genetic studies, participated in the sequence alignment and drafted
%% the manuscript. JY carried out the immunoassays. MT participated in
%% the sequence alignment. ES participated in the design of the study
%% and performed the statistical analysis. FG conceived of the study,
%% and participated in its design and coordination and helped to draft
%% the manuscript. All authors read and approved the final manuscript.

%% All contributors who do not meet the criteria for authorship should
%% be listed in an acknowledgements section. Examples of those who
%% might be acknowledged include a person who provided purely
%% technical help, writing assistance, or a department chair who
%% provided only general support.

\section*{Authors contributions}
    
%%%%%%%%%%%%%%%%%%%%%%%%%%%%%%%%
%% You may choose to use this section to include any relevant
%% information about the author(s) that may aid the reader's
%% interpretation of the article, and understand the standpoint of the
%% author(s). This may include details about the authors'
%% qualifications, current positions they hold at institutions or
%% societies, or any other relevant background information. Please
%% refer to authors using their initials. Note this section should not
%% be used to describe any competing interests.

\section*{Authors information}

%%%%%%%%%%%%%%%%%%%%%%%%%%%
%% Please acknowledge anyone who contributed towards the study by
%% making substantial contributions to conception, design, acquisition
%% of data, or analysis and interpretation of data, or who was
%% involved in drafting the manuscript or revising it critically for
%% important intellectual content, but who does not meet the criteria
%% for authorship. Please also include their source(s) of
%% funding. Please also acknowledge anyone who contributed materials
%% essential for the study.

%% The role of a medical writer must be included in the
%% acknowledgements section, including their source(s) of funding.

%% Authors should obtain permission to acknowledge from all those
%% mentioned in the Acknowledgements.

%% Please list the source(s) of funding for the study, for each
%% author, and for the manuscript preparation in the acknowledgements
%% section. Authors must describe the role of the funding body, if
%% any, in study design; in the collection, analysis, and
%% interpretation of data; in the writing of the manuscript; and in
%% the decision to submit the manuscript for publication.

\section*{Acknowledgements and Funding}
  \ifthenelse{\boolean{publ}}{\small}{}
 
%%%%%%%%%%%%%%%%%%%%%%%%%%%%%%%%%%%%%%%%%%%%%%%%%%%%%%%%%%%%%
%%                  The Bibliography                       %%
%%                                                         %%              
%%  Bmc_article.bst  will be used to                       %%
%%  create a .BBL file for submission, which includes      %%
%%  XML structured for BMC.                                %%
%%                                                         %%
%%                                                         %%
%%  Note that the displayed Bibliography will not          %% 
%%  necessarily be rendered by Latex exactly as specified  %%
%%  in the online Instructions for Authors.                %% 
%%                                                         %%
%%%%%%%%%%%%%%%%%%%%%%%%%%%%%%%%%%%%%%%%%%%%%%%%%%%%%%%%%%%%%


{\ifthenelse{\boolean{publ}}{\footnotesize}{\small}
 \bibliographystyle{bmc_article}  % Style BST file
  \bibliography{bibliography} }     % Bibliography file (usually '*.bib' ) 

%%%%%%%%%%%

\ifthenelse{\boolean{publ}}{\end{multicols}}{}

%%%%%%%%%%%%%%%%%%%%%%%%%%%%%%%%%%%
%%                               %%
%% Figures                       %%
%%                               %%
%% NB: this is for captions and  %%
%% Titles. All graphics must be  %%
%% submitted separately and NOT  %%
%% included in the Tex document  %%
%%                               %%
%%%%%%%%%%%%%%%%%%%%%%%%%%%%%%%%%%%

%%
%% Do not use \listoffigures as most will included as separate files

\section*{Figures}
  \subsection*{Figure 1 - Sample figure title}
      A short description of the figure content
      should go here.

  \subsection*{Figure 2 - Sample figure title}
      Figure legend text.



%%%%%%%%%%%%%%%%%%%%%%%%%%%%%%%%%%%
%%                               %%
%% Tables                        %%
%%                               %%
%%%%%%%%%%%%%%%%%%%%%%%%%%%%%%%%%%%

%% Use of \listoftables is discouraged.
%%
\section*{Tables}
  \subsection*{Table 1 - Sample table title}
    Here is an example of a \emph{small} table in \LaTeX\ using  
    \verb|\tabular{...}|. This is where the description of the table 
    should go. \par \mbox{}
    \par
    \mbox{
      \begin{tabular}{|c|c|c|}
        \hline \multicolumn{3}{|c|}{My Table}\\ \hline
        A1 & B2  & C3 \\ \hline
        A2 & ... & .. \\ \hline
        A3 & ..  & .  \\ \hline
      \end{tabular}
      }
  \subsection*{Table 2 - Sample table title}
    Large tables are attached as separate files but should
    still be described here.



%%%%%%%%%%%%%%%%%%%%%%%%%%%%%%%%%%%
%%                               %%
%% Additional Files              %%
%%                               %%
%%%%%%%%%%%%%%%%%%%%%%%%%%%%%%%%%%%

\section*{Additional Files}
  \subsection*{Additional file 1 --- Sample additional file title}
    Additional file descriptions text (including details of how to
    view the file, if it is in a non-standard format or the file extension).  This might
    refer to a multi-page table or a figure.

  \subsection*{Additional file 2 --- Sample additional file title}
    Additional file descriptions text.






