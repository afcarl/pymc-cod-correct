%%%%%%%%%%%%%%%%%%%%%%%%%%%%%%%%%%%%%%%%%%%%%%
%%                                          %%
%% The Abstract begins here                 %%
%%                                          %%
%% The Section headings here are those for  %%
%% a Research article submitted to a        %%
%% BMC-Series journal.                      %%  
%%                                          %%
%% If your article is not of this type,     %%
%% then refer to the Instructions for       %%
%% authors on http://www.biomedcentral.com  %%
%% and change the section headings          %%
%% accordingly.                             %%   
%%                                          %%
%%%%%%%%%%%%%%%%%%%%%%%%%%%%%%%%%%%%%%%%%%%%%%
%% The abstract of the manuscript should not exceed 350 words and must
%% be structured into separate sections: Background, the context and
%% purpose of the study; Methods, how the study was performed and
%% statistical tests used; Results, the main findings; Conclusions,
%% brief summary and potential implications. Please minimize the use
%% of abbreviations and do not cite references in the abstract; Trial
%% registration, if your research article reports the results of a
%% controlled health care intervention, please list your trial
%% registry, along with the unique identifying number, e.g. Trial
%% registration: Current Controlled Trials ISRCTN73824458. Please note
%% that there should be no space between the letters and numbers of
%% your trial registration number.

\begin{abstract}
        % Do not use inserted blank lines (ie \\) until main body of text.
        \paragraph*{Background:} Text for this section of the abstract. 
      
        \paragraph*{Methods:} Text for this section of the abstract \ldots

        \paragraph*{Results:} Text for this section of the abstract \ldots

        \paragraph*{Conclusions:} Text for this section of the abstract \ldots
\end{abstract}



\ifthenelse{\boolean{publ}}{\begin{multicols}{2}}{}




%%%%%%%%%%%%%%%%%%%%%%%%%%%%%%%%%%%%%%%%%%%%%%
%%                                          %%
%% The Main Body begins here                %%
%%                                          %%
%% The Section headings here are those for  %%
%% a Research article submitted to a        %%
%% BMC-Series journal.                      %%  
%%                                          %%
%% If your article is not of this type,     %%
%% then refer to the instructions for       %%
%% authors on:                              %%
%% http://www.biomedcentral.com/info/authors%%
%% and change the section headings          %%
%% accordingly.                             %% 
%%                                          %%
%% See the Results and Discussion section   %%
%% for details on how to create sub-sections%%
%%                                          %%
%% use \cite{...} to cite references        %%
%%  \cite{koon} and                         %%
%%  \cite{oreg,khar,zvai,xjon,schn,pond}    %%
%%  \nocite{smith,marg,hunn,advi,koha,mouse}%%
%%                                          %%
%%%%%%%%%%%%%%%%%%%%%%%%%%%%%%%%%%%%%%%%%%%%%%

%%%%%%%%%%%%%%%%
%% Background %%
%%
%% The background section should be written from the standpoint of
%% researchers without specialist knowledge in that area and must
%% clearly state - and, if helpful, illustrate - the background to the
%% research and its aims. Reports of clinical research should, where
%% appropriate, include a summary of a search of the literature to
%% indicate why this study was necessary and what it aimed to
%% contribute to the field. The section should end with a very brief
%% statement of what is being reported in the article.

\section*{Background}
  (FIXME: need to talk about GBD here...different models for each cause...different levels in the
  cause hierarchy).  Models for each cause are estimated separately, so there is no guarantee of 
  consistency and the estimated cause fractions for any given country-sex-year do not necessarily 
  sum to one. \pb

  (FIXME: this would be where a review of the literature would be appropriate, had we done one.) \pb

  A simulation environment was constructed to replicate important characteristics of 
  the problem described.  We propose two models that take different approaches to predicting 
  consistent cause fractions from inconsistent preliminary estimates.  These two models are then  
  validated in the simulation environment and compared on a variety of metrics. \pb

%%%%%%%%%%%%%%%%%%
%% Methods
%%
%% This should include the design of the study, the setting, the type
%% of participants or materials involved, a clear description of all
%% interventions and comparisons, and the type of analysis used,
%% including a power calculation if appropriate.

\section*{Methods}
  \subsection*{Simulation Environment}
     Simulated cause fraction estimates are generated for 3 causes over a period of 4 years. 
    (FIXME: the number of causes and the length of the time period could change). 'True' 
    cause fractions are specified for each of the causes in each of the years. Additionally, 
    a 'true' precision is specified for each cause fraction, and this is treated as 
    constant over time (FIXME: should probably explain what this precision is supposed to 
    be telling us about the cause fractions). A single draw from a normal distribution 
    with mean 0 and precision equal to the true cause fraction precision is generated. 
    This is added to the true cause fractions to imitate an unbiased preliminary estimate of the
    true cause fractions. A set of 1000 draws from a normal distribution using this estimated
    cause fraction as the mean and the true precision as the precision is taken as the
    'data' for this simulation that will eventually be used in the modeling process. 
    This procedure is repeated, resulting in 50 sets of simulated preliminary 
    estimates for each specified set of cause fractions. All draws from normal distributions 
    are carried out in logit space to guarantee that the resulting cause fractions are each
    between zero and one. Both of the models described below are fit on each of the 50 sets 
    of preliminary estimates for each specified cause fraction scenario. \pb

    A variety of different scenarios are tested (FIXME: must decide on what the scenarios are
    before I can describe them) \pb

  \subsection*{Statistical Model}
    Two models are proposed for estimating consistent cause fractions. The first takes an obvious
    approach: all cause fractions are simply scaled by a factor equal to their sum: 
    (FIXME: need to put some math here, i.e. the bad model)
  \begin{align*}
    math!
  \end{align*}
    For the simulated estimates described above, this is carried out individually for 
    each simulation and then the mean estimate for each cause is taken as the final 
    estimate for each cause fraction. The 95\% high probability density region (95\% hpd) 
    is also calculated and used to generate estimates of the upper and lower bounds of 
    the final cause fraction estimates. \pb

    The second model proposed attempts to take into account differences in the estimated 
    precision of the preliminary cause fraction estimates. Additionally, it explicitly 
    takes into account correlation in these preliminary estimates over time (FIXME: this 
    may not actually be true. Must understand the model better before writing about it). 
    (FIXME: probably need a more thorough verbal description here of the model; also 
    need to update the formula Abie typed below). 
  \begin{align*}
    \Pr[Y | \pi, \tau] &= \sum_{i=1}^N \frac{1}{N} \Pr[Y_i | \pi, \tau]\\
    Y_i | \pi, \tau &\sim N(\pi, \tau)\\
    \pi &\sim \Dirichlet(1) \\
    \tau &\sim \Uniform[0, \infty) \\
  \end{align*}
    (FIXME: need to talk about how the model is fit). The mean of pi across all retained 
    iterations of the MCMC chain is taken as the final estimate of the cause fractions 
    and the 95\% hpd is calculated from all retained iterations. \pb

  \subsection*{Quality Metrics}
    The final cause fraction estimates from each model are compared to the known true cause 
    fractions and the absolute error, relative error, and coverage (i.e. the true estimate
    falls between the upper and lower bounds) are calculated for each cause and each 
    of the 50 sets of estimates. The mean and median absolute error, mean and 
    median relative error, and mean coverage are estimated by cause across all sets of 
    estimates. Mean coverage is also estimated for all causes and the 
    percent of times across all 50 sets of estimates that all causes are covered (full 
    coverage) is also computed. One additional metric, cause specific mortality fraction 
    accuracy (CSMF accuracy) is also computed for each of the 50 sets of estimates 
    according to the following formula: 
  \begin{align*}
    1-\sum \left | CSMF_{pred} - CSMF_{true} \right |/(2*(1-min(cf_{true})))
  \end{align*}
    The mean and median CSMF accuracy are then computed across all sets of estimates. \pb 

    (FIXME: need to remove some of the metrics here that we aren't ultimately going to 
    use and also need to describe why each one of the ones that remain are interesting 
    and necessary given the others) \pb 

%%%%%%%%%%%%%%%%%%%%%%%%%%%%
%% Results  %%
%%
%% The Results and Discussion may be combined into a single section or
%% presented separately. Results of statistical analysis should
%% include, where appropriate, relative and absolute risks or risk
%% reductions, and confidence intervals. The results and discussion
%% sections may also be broken into subsections with short,
%% informative headings.

\section*{Results/Discussion}
  \subsection*{Results sub-heading}

  \subsection*{Another results sub-heading}

  \subsection*{Yet another results sub-heading}

%%%%%%%%%%%%%%%%%%%%%%
%% This should state clearly the main conclusions of the research and
%% give a clear explanation of their importance and relevance. Summary
%% illustrations may be included.

\section*{Conclusions}
The good model is better than the bad model! Hurray!!! \pb
  
%%%%%%%%%%%%%%%%%%%%%%
%% If abbreviations are used in the text, either they should be
%% defined in the text where first used, or a list of abbreviations
%% can be provided, which should precede the competing interests and
%% authors' contributions.

\section*{List of abbreviations used}
    
%%%%%%%%%%%%%%%%%%%%%%%%%%%%%%%%
%% A competing interest exists when your interpretation of data or
%% presentation of information may be influenced by your personal or
%% financial relationship with other people or organizations. Authors
%% should disclose any financial competing interests but also any
%% non-financial competing interests that may cause them embarrassment
%% were they to become public after the publication of the manuscript.
%%
%% Authors are required to complete a declaration of competing
%% interests. All competing interests that are declared will be listed
%% at the end of published articles. Where an author gives no
%% competing interests, the listing will read 'The author(s) declare
%% that they have no competing interests'.
%%
%% When completing your declaration, please consider the following
%% questions:
%%
%% Financial competing interests
%%
%% * In the past five years have you received reimbursements, fees,
%%   funding, or salary from an organization that may in any way gain or
%%   lose financially from the publication of this manuscript, either
%%   now or in the future? Is such an organization financing this
%%   manuscript (including the article-processing charge)? If so, please
%%   specify.
%% * Do you hold any stocks or shares in an organization that may in
%%   any way gain or lose financially from the publication of this
%%   manuscript, either now or in the future? If so, please specify.
%% * Do you hold or are you currently applying for any patents
%%   relating to the content of the manuscript? Have you received
%%   reimbursements, fees, funding, or salary from an organization
%%   that holds or has applied for patents relating to the content of
%%   the manuscript? If so, please specify.
%% * Do you have any other financial competing interests? If so,
%%   please specify.
%%
%% Non-financial competing interests
%%
%% * Are there any non-financial competing interests (political,
%%   personal, religious, ideological, academic, intellectual,
%%   commercial or any other) to declare in relation to this
%%   manuscript? If so, please specify.
%%
%% If you are unsure as to whether you or one of your co-authors has a
%% competing interest, please discuss it with the editorial office.

\section*{Competing interests }

%%%%%%%%%%%%%%%%%%%%%%%%%%%%%%%%
%% 
%% In order to give appropriate credit to each author of a paper, the
%% individual contributions of authors to the manuscript should be
%% specified in this section.

%% An ``author'' is generally considered to be someone who has made
%% substantive intellectual contributions to a published study. To
%% qualify as an author one should 1) have made substantial
%% contributions to conception and design, or acquisition of data, or
%% analysis and interpretation of data; 2) have been involved in
%% drafting the manuscript or revising it critically for important
%% intellectual content; and 3) have given final approval of the
%% version to be published. Each author should have participated
%% sufficiently in the work to take public responsibility for
%% appropriate portions of the content. Acquisition of funding,
%% collection of data, or general supervision of the research group,
%% alone, does not justify authorship.

%% We suggest the following kind of format (please use initials to
%% refer to each author's contribution): AB carried out the molecular
%% genetic studies, participated in the sequence alignment and drafted
%% the manuscript. JY carried out the immunoassays. MT participated in
%% the sequence alignment. ES participated in the design of the study
%% and performed the statistical analysis. FG conceived of the study,
%% and participated in its design and coordination and helped to draft
%% the manuscript. All authors read and approved the final manuscript.

%% All contributors who do not meet the criteria for authorship should
%% be listed in an acknowledgements section. Examples of those who
%% might be acknowledged include a person who provided purely
%% technical help, writing assistance, or a department chair who
%% provided only general support.

\section*{Authors contributions}
    
%%%%%%%%%%%%%%%%%%%%%%%%%%%%%%%%
%% You may choose to use this section to include any relevant
%% information about the author(s) that may aid the reader's
%% interpretation of the article, and understand the standpoint of the
%% author(s). This may include details about the authors'
%% qualifications, current positions they hold at institutions or
%% societies, or any other relevant background information. Please
%% refer to authors using their initials. Note this section should not
%% be used to describe any competing interests.

\section*{Authors information}

%%%%%%%%%%%%%%%%%%%%%%%%%%%
%% Please acknowledge anyone who contributed towards the study by
%% making substantial contributions to conception, design, acquisition
%% of data, or analysis and interpretation of data, or who was
%% involved in drafting the manuscript or revising it critically for
%% important intellectual content, but who does not meet the criteria
%% for authorship. Please also include their source(s) of
%% funding. Please also acknowledge anyone who contributed materials
%% essential for the study.

%% The role of a medical writer must be included in the
%% acknowledgements section, including their source(s) of funding.

%% Authors should obtain permission to acknowledge from all those
%% mentioned in the Acknowledgements.

%% Please list the source(s) of funding for the study, for each
%% author, and for the manuscript preparation in the acknowledgements
%% section. Authors must describe the role of the funding body, if
%% any, in study design; in the collection, analysis, and
%% interpretation of data; in the writing of the manuscript; and in
%% the decision to submit the manuscript for publication.

\section*{Acknowledgements and Funding}
  \ifthenelse{\boolean{publ}}{\small}{}
 
%%%%%%%%%%%%%%%%%%%%%%%%%%%%%%%%%%%%%%%%%%%%%%%%%%%%%%%%%%%%%
%%                  The Bibliography                       %%
%%                                                         %%              
%%  Bmc_article.bst  will be used to                       %%
%%  create a .BBL file for submission, which includes      %%
%%  XML structured for BMC.                                %%
%%                                                         %%
%%                                                         %%
%%  Note that the displayed Bibliography will not          %% 
%%  necessarily be rendered by Latex exactly as specified  %%
%%  in the online Instructions for Authors.                %% 
%%                                                         %%
%%%%%%%%%%%%%%%%%%%%%%%%%%%%%%%%%%%%%%%%%%%%%%%%%%%%%%%%%%%%%


{\ifthenelse{\boolean{publ}}{\footnotesize}{\small}
 \bibliographystyle{bmc_article}  % Style BST file
  \bibliography{bibliography} }     % Bibliography file (usually '*.bib' ) 

%%%%%%%%%%%

\ifthenelse{\boolean{publ}}{\end{multicols}}{}

%%%%%%%%%%%%%%%%%%%%%%%%%%%%%%%%%%%
%%                               %%
%% Figures                       %%
%%                               %%
%% NB: this is for captions and  %%
%% Titles. All graphics must be  %%
%% submitted separately and NOT  %%
%% included in the Tex document  %%
%%                               %%
%%%%%%%%%%%%%%%%%%%%%%%%%%%%%%%%%%%

%%
%% Do not use \listoffigures as most will included as separate files

\section*{Figures}
  \subsection*{Figure 1 - Sample figure title}
      A short description of the figure content
      should go here.

  \subsection*{Figure 2 - Sample figure title}
      Figure legend text.



%%%%%%%%%%%%%%%%%%%%%%%%%%%%%%%%%%%
%%                               %%
%% Tables                        %%
%%                               %%
%%%%%%%%%%%%%%%%%%%%%%%%%%%%%%%%%%%

%% Use of \listoftables is discouraged.
%%
\section*{Tables}
  \subsection*{Table 1 - Sample table title}
    Here is an example of a \emph{small} table in \LaTeX\ using  
    \verb|\tabular{...}|. This is where the description of the table 
    should go. \par \mbox{}
    \par
    \mbox{
      \begin{tabular}{|c|c|c|}
        \hline \multicolumn{3}{|c|}{My Table}\\ \hline
        A1 & B2  & C3 \\ \hline
        A2 & ... & .. \\ \hline
        A3 & ..  & .  \\ \hline
      \end{tabular}
      }
  \subsection*{Table 2 - Sample table title}
    Large tables are attached as separate files but should
    still be described here.



%%%%%%%%%%%%%%%%%%%%%%%%%%%%%%%%%%%
%%                               %%
%% Additional Files              %%
%%                               %%
%%%%%%%%%%%%%%%%%%%%%%%%%%%%%%%%%%%

\section*{Additional Files}
  \subsection*{Additional file 1 --- Sample additional file title}
    Additional file descriptions text (including details of how to
    view the file, if it is in a non-standard format or the file extension).  This might
    refer to a multi-page table or a figure.

  \subsection*{Additional file 2 --- Sample additional file title}
    Additional file descriptions text.






